%%% Two notes on column specification:
%%% (i)  Column widths are manually chosen as small as possible to allow
%%%      for a wider last X column.
%%% (ii)  In the first column \linepenalty=100 prefers shorter paragraphs
%%%       (less lines), where plain \raggedright were indifferent and
%%%       sometimes caused a dangling line, e.g., for 'directionpoint of'
%%%       or 'directiontime of'.
\begin{longtable}{|>{\raggedright\linepenalty=100\ttfamily}p{.793in}*{3}{|>{\raggedright}p{.715in}}|>{\raggedleft}p{1.5em}|>{\raggedright\arraybackslash}X|}
\caption{\strut Операторы}\label{optab}\\
%%\caption{\strut Operators}\label{optab}\\
\hline
Имя&  \multicolumn3{c|}{Аргумент/типы результата}&  \makebox[.2in][c]{Стр.}&  Объяснение\\\cline{2-4}
%%Name&  \multicolumn3{c|}{Argument/result types}&  \makebox[.2in][c]{Page}&  Explanation\\\cline{2-4}
&  \multicolumn1{c|}{Левый}&  \multicolumn1{c|}{Правый}&  \multicolumn1{c|}{Результат}&  &  \\
%%&  \multicolumn1{c|}{Left}&  \multicolumn1{c|}{Right}&  \multicolumn1{c|}{Result}&  &  \\
\hline
\hline
\endfirsthead
\caption[]{\strut Операторы \emph{(продолжение)}}\\
%%\caption[]{\strut Operators \emph{(continued)}}\\
\hline
Имя&  \multicolumn3{c|}{Аргумент/типы результата}&  \makebox[.2in][c]{Стр.}&  Объяснение\\\cline{2-4}
%%Name&  \multicolumn3{c|}{Argument/result types}&  \makebox[.2in][c]{Page}&  Explanation\\\cline{2-4}
&  \multicolumn1{c|}{Левый}&  \multicolumn1{c|}{Правый}&  \multicolumn1{c|}{Результат}&  &  \\
%%&  \multicolumn1{c|}{Left}&  \multicolumn1{c|}{Right}&  \multicolumn1{c|}{Result}&  &  \\
\hline
\hline
\endhead
\&\index{&?\texttt{\&}}&  string\par path&  string\par path &  string\par path&  \pageref{Damp}&  Склейка (для путей $l\hbox{\tt\&}r$, если $r$ начинается точно там, где кончается $l$)\\\hline
%%\&\index{&?\texttt{\&}}&  string\par path&  string\par path &  string\par path&  \pageref{Damp}&  Concatenation---works for paths $l\hbox{\tt\&}r$ if $r$ starts exactly where the $l$ ends\\\hline
*\index{*?\texttt{*}}&  numeric&  (cmyk)color\par numeric\par pair&  (cmyk)color\par numeric\par pair&  \pageref{Dmldiv}&  Умножение\\\hline
%%*\index{*?\texttt{*}}&  numeric&  (cmyk)color\par numeric\par pair&  (cmyk)color\par numeric\par pair&  \pageref{Dmldiv}&  Multiplication\\\hline
*\index{*?\texttt{*}}&  (cmyk)color\par numeric\par pair&  numeric&  (cmyk)color\par numeric\par pair&  \pageref{Dmldiv}&  Умножение\\\hline
%%*\index{*?\texttt{*}}&  (cmyk)color\par numeric\par pair&  numeric&  (cmyk)color\par numeric\par pair&  \pageref{Dmldiv}&  Multiplication\\\hline
**\index{**?\texttt{**}}&  numeric&  numeric&  numeric&  \pageref{Dpow}&  Возведение в степень\\\hline
%%**\index{**?\texttt{**}}&  numeric&  numeric&  numeric&  \pageref{Dpow}&  Exponentiation\\\hline
+\index{+?\texttt{+}}&  (cmyk)color\par numeric\par pair&  (cmyk)color\par numeric\par pair&  (cmyk)color\par numeric\par pair&  \pageref{Dadd}&  Сложение\\\hline
%%+\index{+?\texttt{+}}&  (cmyk)color\par numeric\par pair&  (cmyk)color\par numeric\par pair&  (cmyk)color\par numeric\par pair&  \pageref{Dadd}&  Addition\\\hline
++\index{++?\texttt{++}}&  numeric&  numeric&  numeric&  \pageref{Dpyadd}&  Пифагорово сложение $\sqrt{l^2+r^2}$\\\hline
%%++\index{++?\texttt{++}}&  numeric&  numeric&  numeric&  \pageref{Dpyadd}&  Pythagorean addition $\sqrt{l^2+r^2}$\\\hline
+-+\index{+-+?\texttt{+-+}}&  numeric&  numeric&  numeric&  \pageref{Dpysub}&  Пифагорово вычитание $\sqrt{l^2-r^2}$\\\hline
%%+-+\index{+-+?\texttt{+-+}}&  numeric&  numeric&  numeric&  \pageref{Dpysub}&  Pythagorean subtraction $\sqrt{l^2-r^2}$\\\hline
-\index{-?\texttt{-}}&  (cmyk)color\par numeric\par pair&  (cmyk)color\par numeric\par pair&  (cmyk)color\par numeric\par pair&  \pageref{Dadd}&  Вычитание\\\hline
%%-\index{-?\texttt{-}}&  (cmyk)color\par numeric\par pair&  (cmyk)color\par numeric\par pair&  (cmyk)color\par numeric\par pair&  \pageref{Dadd}&  Subtraction\\\hline
-\index{-?\texttt{-}}&  --&  (cmyk)color\par numeric\par pair&  (cmyk)color\par numeric\par pair&  \pageref{Dneg}&  Унарный минус\\\hline
%%-\index{-?\texttt{-}}&  --&  (cmyk)color\par numeric\par pair&  (cmyk)color\par numeric\par pair&  \pageref{Dneg}&  Negation\\\hline
/\index{/?\texttt{/}}&  (cmyk)color\par numeric\par pair&  numeric&  (cmyk)color\par numeric\par pair&  \pageref{Dmldiv}&  Деление\\\hline
%%/\index{/?\texttt{/}}&  (cmyk)color\par numeric\par pair&  numeric&  (cmyk)color\par numeric\par pair&  \pageref{Dmldiv}&  Division\\\hline
<\index{<?\texttt{<}} =\index{=?\texttt{=}} >>\index{>?\texttt{>}}\par <=\index{<=?\texttt{<=}} >=\index{=>?\texttt{=>}}\par <>\index{<>?\texttt{<>}}&  string\par numeric\par pair\par (cmyk)color\par transform&  string\par numeric\par pair\par (cmyk)color\par transform&  boolean&  \pageref{Dcmpar}&  Операции сравнения\\\hline
%%<\index{<?\texttt{<}} =\index{=?\texttt{=}} >>\index{>?\texttt{>}}\par <=\index{<=?\texttt{<=}} >=\index{=>?\texttt{=>}}\par <>\index{<>?\texttt{<>}}&  string\par numeric\par pair\par (cmyk)color\par transform&  string\par numeric\par pair\par (cmyk)color\par transform&  boolean&  \pageref{Dcmpar}&  Comparison operators\\\hline
\pl abs\index{abs?\texttt{abs}}&  --&  numeric\par pair&  numeric&  \pageref{Dabs}&  Модуль\par Евклидова длина $\sqrt{(\mbox{\ttfamily xpart\ } r)^2+(\mbox{\ttfamily ypart\ } r)^2}$\\\hline
%%\pl abs\index{abs?\texttt{abs}}&  --&  numeric\par pair&  numeric&  \pageref{Dabs}&  Absolute value\par Euclidean length $\sqrt{(\mbox{\ttfamily xpart\ } r)^2+(\mbox{\ttfamily ypart\ } r)^2}$\\\hline
and\index{and?\texttt{and}}&  boolean&  boolean&  boolean&  \pageref{Dand}&  Логическое И\\\hline
%%and\index{and?\texttt{and}}&  boolean&  boolean&  boolean&  \pageref{Dand}&  Logical and\\\hline
angle\index{angle?\texttt{angle}}&  --&  pair&  numeric&  \pageref{Dangle}&  2$-$аргументный арктангенс (в градусах)\\\hline
%%angle\index{angle?\texttt{angle}}&  --&  pair&  numeric&  \pageref{Dangle}&  2$-$argument arctangent (in degrees)\\\hline
arclength\index{arclength?\texttt{arclength}}&  --&  path&  numeric&  \pageref{Darclng}&  Длина дуги пути\\\hline
%%arclength\index{arclength?\texttt{arclength}}&  --&  path&  numeric&  \pageref{Darclng}&  Arc length of a path\\\hline
arctime of\index{arctime of?\texttt{arctime of}}&  numeric&  path&  numeric&  \pageref{Darctim}&  Время на пути, где длина дуги от начала достигает заданной величины\\\hline
%%arctime of\index{arctime of?\texttt{arctime of}}&  numeric&  path&  numeric&  \pageref{Darctim}&  Time on a path where arc length from the start reaches a given value\\\hline
ASCII\index{ASCII?\texttt{ASCII}}&  --&  string&  numeric&  --&  Значение ASCII первого символа в строке\\\hline
%%ASCII\index{ASCII?\texttt{ASCII}}&  --&  string&  numeric&  --&  ASCII value of first character in string\\\hline
\pl bbox\index{bbox?\texttt{bbox}}&  --&  picture\par path\par pen&  path&  \pageref{Dbbox}&  Прямоугольный путь охватывающей рамки\\\hline
%%\pl bbox\index{bbox?\texttt{bbox}}&  --&  picture\par path\par pen&  path&  \pageref{Dbbox}&  A rectangular path for the bounding box\\\hline
blackpart\index{blackpart?\texttt{blackpart}}&  --&  cmykcolor&  numeric&  \pageref{Dcmykprt}&  Выделение четвертой компоненты\\\hline
%%blackpart\index{blackpart?\texttt{blackpart}}&  --&  cmykcolor&  numeric&  \pageref{Dcmykprt}&  Extract the fourth component\\\hline
bluepart\index{bluepart?\texttt{bluepart}}&  --&  color&  numeric&  \pageref{Drgbprt}&  Выделение третьей компоненты\\\hline
%%bluepart\index{bluepart?\texttt{bluepart}}&  --&  color&  numeric&  \pageref{Drgbprt}&  Extract the third component\\\hline
boolean\index{boolean?\texttt{boolean}}&  --&  любой&  boolean&  \pageref{Dboolop}&  Выражение логического типа?\\\hline
%%boolean\index{boolean?\texttt{boolean}}&  --&  any&  boolean&  \pageref{Dboolop}&  Is the expression of type boolean?\\\hline
\pl bot\index{bot?\texttt{bot}}&  --&  numeric\par pair&  numeric\par pair&  \pageref{Dbot}&  Низ текущего пера, центрированного по заданным координатам\\\hline
%%\pl bot\index{bot?\texttt{bot}}&  --&  numeric\par pair&  numeric\par pair&  \pageref{Dbot}&  Bottom of current pen when centered at the given coordinate(s)\\\hline
bounded\index{bounded?\texttt{bounded}}&  --&  любой&  boolean&  \pageref{Dbounded}&  Аргумент --- это картинка в охватывающей рамке?\\\hline
%%bounded\index{bounded?\texttt{bounded}}&  --&  any&  boolean&  \pageref{Dbounded}&  Is argument a picture with a bounding box?\\\hline
\pl ceiling\index{ceiling?\texttt{ceiling}}&  --&  numeric&  numeric&  \pageref{Dceil}&  Наименьшее целое, большее или равное данному\\\hline
%%\pl ceiling\index{ceiling?\texttt{ceiling}}&  --&  numeric&  numeric&  \pageref{Dceil}&  Least integer greater than or equal to\\\hline
\pl center\index{center?\texttt{center}}&  --&  picture\par path\par pen&  pair&  \pageref{Dcenter}&  Центр охватывающей рамки\\\hline
%%\pl center\index{center?\texttt{center}}&  --&  picture\par path\par pen&  pair&  \pageref{Dcenter}&  Center of the bounding box\\\hline
char\index{char?\texttt{char}}&  --&  numeric&  string&  \pageref{Dchar}&  Символ с заданным кодом ASCII\\\hline
%%char\index{char?\texttt{char}}&  --&  numeric&  string&  \pageref{Dchar}&  Character with a given ASCII code\\\hline
clipped\index{clipped?\texttt{clipped}}&  --&  любой&  boolean&  \pageref{Dclipped}&  Аргумент --- вырезка из картинки?\\\hline
%%clipped\index{clipped?\texttt{clipped}}&  --&  any&  boolean&  \pageref{Dclipped}&  Is argument a clipped picture?\\\hline
cmykcolor\index{cmykcolor?\texttt{cmykcolor}}&  --&  любой&  boolean&  \pageref{Dccolrop}&  Выражение типа cmyk-цвет?\\\hline
%%cmykcolor\index{cmykcolor?\texttt{cmykcolor}}&  --&  any&  boolean&  \pageref{Dccolrop}&  Is the expression of type cmykcolor?\\\hline
color\index{color?\texttt{color}}&  --&  любой&  boolean&  \pageref{Dcolrop}&  Выражение типа цвет?\\\hline
%%color\index{color?\texttt{color}}&  --&  any&  boolean&  \pageref{Dcolrop}&  Is the expression of type color?\\\hline
colormodel\index{colormodel?\texttt{colormodel}}&  --&  изображение&  numeric&  \pageref{Dcolormodel}&  Какая модель цвета в объекте-изображении?\\\hline
%%colormodel\index{colormodel?\texttt{colormodel}}&  --&  image object&  numeric&  \pageref{Dcolormodel}&  What is the color model of the image object?\\\hline
\pl colorpart\index{colorpart?\texttt{colorpart}}&  --&  изображение&
(cmyk)color\par numeric\par boolean&  \pageref{Dcolorpart}&  Каков цвет 
объекта-изображения?\\\hline
%%\pl colorpart\index{colorpart?\texttt{colorpart}}&  --&  image object&
%%(cmyk)color\par numeric\par boolean&  \pageref{Dcolorpart}&  What is the
%%color of the image object?\\\hline
cosd\index{cosd?\texttt{cosd}}&  --&  numeric&  numeric&  \pageref{Dcosd}&  Косинус угла в градусах\\\hline
%%cosd\index{cosd?\texttt{cosd}}&  --&  numeric&  numeric&  \pageref{Dcosd}&  Cosine of angle in degrees\\\hline
\pl cutafter\index{cutafter?\texttt{cutafter}}&  path&  path&  path&  \pageref{Dcuta}&  Левый аргумент с частью, отбрасываемой после пересечения\\\hline
%%\pl cutafter\index{cutafter?\texttt{cutafter}}&  path&  path&  path&  \pageref{Dcuta}&  Left argument with part after the intersection dropped\\\hline
\pl cutbefore\index{cutbefore?\texttt{cutbefore}}&  path&  path&  path&  \pageref{Dcutb}&  Левый аргумент с частью, отбрасываемой до пересечения\\\hline
%%\pl cutbefore\index{cutbefore?\texttt{cutbefore}}&  path&  path&  path&  \pageref{Dcutb}&  Left argument with part before the intersection dropped\\\hline
cyanpart\index{cyanpart?\texttt{cyanpart}}&  --&  cmykcolor&  numeric&  \pageref{Dcmykprt}&  Извлечь первый аргумент\\\hline
%%cyanpart\index{cyanpart?\texttt{cyanpart}}&  --&  cmykcolor&  numeric&  \pageref{Dcmykprt}&  Extract the first component\\\hline
cycle\index{cycle?\texttt{cycle}}&  --&  path&  boolean&  \pageref{Dcycop}&  Определяет цикличен ли путь\\\hline
%%cycle\index{cycle?\texttt{cycle}}&  --&  path&  boolean&  \pageref{Dcycop}&  Determines whether a path is cyclic\\\hline
dashpart\index{dashpart?\texttt{dashpart}}&  --&  picture&  picture&  \pageref{Ddashpart}&  Образец пунктира пути в рисуемой картинке\\\hline
%%dashpart\index{dashpart?\texttt{dashpart}}&  --&  picture&  picture&  \pageref{Ddashpart}&  Dash pattern of a path in a stroked picture\\\hline
decimal\index{decimal?\texttt{decimal}}&  --&  numeric&  string&  \pageref{Ddecop}&  Десятичное представление\\\hline
%%decimal\index{decimal?\texttt{decimal}}&  --&  numeric&  string&  \pageref{Ddecop}&  The decimal representation\\\hline
\pl dir\index{dir?\texttt{dir}}&  --&  numeric&  pair&  \pageref{Ddirop}&  $(\cos\theta,\sin\theta)$ по заданному $\theta$ в градусах\\\hline
%%\pl dir\index{dir?\texttt{dir}}&  --&  numeric&  pair&  \pageref{Ddirop}&  $(\cos\theta,\sin\theta)$ given $\theta$ in degrees\\\hline
\pl direction of\index{direction of?\texttt{direction of}}&  numeric&  path&  pair&  \pageref{Ddirof}&  Направление пути в данное `время'\\\hline
%%\pl direction of\index{direction of?\texttt{direction of}}&  numeric&  path&  pair&  \pageref{Ddirof}&  The direction of a path at a given `time'\\\hline
\pl direction\-point of\index{directionpoint of?\texttt{directionpoint of}}&  pair&  path&  numeric&  \pageref{Ddpntof}&  Точка, где путь имеет заданное направление\\\hline
%%\pl direction\-point of\index{directionpoint of?\texttt{directionpoint of}}&  pair&  path&  numeric&  \pageref{Ddpntof}&  Point where a path has a given direction\\\hline
direction\-time of\index{directiontime of?\texttt{directiontime of}}&  pair&  path&  numeric&  \pageref{Ddtimof}&  `Время', когда путь имеет заданное направление\\\hline
%%direction\-time of\index{directiontime of?\texttt{directiontime of}}&  pair&  path&  numeric&  \pageref{Ddtimof}&  `Time' when a path has a given direction\\\hline
\pl div\index{div?\texttt{div}}&  numeric&  numeric&  numeric&  --&  Целочисленное деление $\lfloor l/r\rfloor$\\\hline
%%\pl div\index{div?\texttt{div}}&  numeric&  numeric&  numeric&  --&  Integer division $\lfloor l/r\rfloor$\\\hline
\pl dotprod\index{dotprod?\texttt{dotprod}}&  pair&  pair&  numeric&  \pageref{Ddprod}&  скалярное произведение векторов\\\hline
%%\pl dotprod\index{dotprod?\texttt{dotprod}}&  pair&  pair&  numeric&  \pageref{Ddprod}&  vector dot product\\\hline
filled\index{filled?\texttt{filled}}&  --&  любой&  boolean&  \pageref{Dfilled}&  Аргумент --- это заполненное выделение?\\\hline
%%filled\index{filled?\texttt{filled}}&  --&  any&  boolean&  \pageref{Dfilled}&  Is argument a filled outline?\\\hline
floor\index{floor?\texttt{floor}}&  --&  numeric&  numeric&  \pageref{Dfloor}&  Наибольшее целое, меньшее или равное данному\\\hline
%%floor\index{floor?\texttt{floor}}&  --&  numeric&  numeric&  \pageref{Dfloor}&  Greatest integer less than or equal to\\\hline
fontpart\index{fontpart?\texttt{fontpart}}&  --&  picture&  string&  \pageref{Dfontpart}&  Шрифт текстовой компоненты картинки\\\hline
%%fontpart\index{fontpart?\texttt{fontpart}}&  --&  picture&  string&  \pageref{Dfontpart}&  Font of a textual picture component\\\hline
fontsize\index{fontsize?\texttt{fontsize}}&  --&  string&  numeric&  \pageref{Dfntsiz}&  Размер шрифта в пунктах\\\hline
%%fontsize\index{fontsize?\texttt{fontsize}}&  --&  string&  numeric&  \pageref{Dfntsiz}&  The point size of a font\\\hline
greenpart\index{greenpart?\texttt{greenpart}}&  --&  color&  numeric&  \pageref{Drgbprt}&  Выделить второй компонент\\\hline
%%greenpart\index{greenpart?\texttt{greenpart}}&  --&  color&  numeric&  \pageref{Drgbprt}&  Extract the second component\\\hline
greypart\index{greypart?\texttt{greypart}}&  --&  numeric&  numeric&   \pageref{Dgreyprt}&  Выделить первый (единственный) компонент\\\hline
%%greypart\index{greypart?\texttt{greypart}}&  --&  numeric&  numeric&   \pageref{Dgreyprt}&  Extract the first (only) component\\\hline
hex\index{hex?\texttt{hex}}&  --&  string&  numeric&  --&  Интерпретировать как 16-ричное число\\\hline
%%hex\index{hex?\texttt{hex}}&  --&  string&  numeric&  --&  Interpret as a hexadecimal number\\\hline
infont\index{infont?\texttt{infont}}&  string&  string&  picture&  \pageref{Sinfont}&  Печать строку в заданном шрифте\\\hline
%%infont\index{infont?\texttt{infont}}&  string&  string&  picture&  \pageref{Sinfont}&  Typeset string in given font\\\hline
\pl intersec\-tionpoint\index{intersectionpoint?\texttt{intersectionpoint}}&  path&  path&  pair&  \pageref{Disecpt}&  Точка пересечения\\\hline
%%\pl intersec\-tionpoint\index{intersectionpoint?\texttt{intersectionpoint}}&  path&  path&  pair&  \pageref{Disecpt}&  An intersection point\\\hline
intersec\-tiontimes\index{intersectiontimes?\texttt{intersectiontimes}}&  path&  path&  pair&  \pageref{Disectt}&  Времена ($t_l,t_r)$ на путях $l$ и $r$, когда пути пересекаются\\\hline
%%intersec\-tiontimes\index{intersectiontimes?\texttt{intersectiontimes}}&  path&  path&  pair&  \pageref{Disectt}&  Times ($t_l,t_r)$ on paths $l$ and $r$ when the paths intersect\\\hline
\pl inverse\index{inverse?\texttt{inverse}}&  --&  transform&  transform&  \pageref{Dinv}&  Обратить трансформацию\\\hline
%%\pl inverse\index{inverse?\texttt{inverse}}&  --&  transform&  transform&  \pageref{Dinv}&  Invert a transformation\\\hline
known\index{known?\texttt{known}}&  --&  любой&  boolean&  \pageref{Dknown}&  Имеет ли аргумент известное значение?\\\hline
%%known\index{known?\texttt{known}}&  --&  any&  boolean&  \pageref{Dknown}&  Does argument have a known value?\\\hline
length\index{length?\texttt{length}}&  --&  path\par string\par picture&  numeric&  \pageref{Dlength}\par \pageref{DlengthString}\par \pageref{DlengthPicture}&  Число компонент (дуг, символов, нарисованных объектов, \ldots) в аргументе\\\hline
%%length\index{length?\texttt{length}}&  --&  path\par string\par picture&  numeric&  \pageref{Dlength}\par \pageref{DlengthString}\par \pageref{DlengthPicture}&  Number of components (arcs, characters, strokes, \ldots) in the argument\\\hline
\pl lft\index{lft?\texttt{lft}}&  --&  numeric\par pair&  numeric\par pair&  \pageref{Dlft}&  Левый край текущего пера с центром по заданным координатам\\\hline
%%\pl lft\index{lft?\texttt{lft}}&  --&  numeric\par pair&  numeric\par pair&  \pageref{Dlft}&  Left side of current pen when its center is at the given coordinate(s)\\\hline
llcorner\index{llcorner?\texttt{llcorner}}&  --&  picture\par path\par pen&  pair&  \pageref{Dcornop}&  Нижний левый угол охватывающей рамки\\\hline
%%llcorner\index{llcorner?\texttt{llcorner}}&  --&  picture\par path\par pen&  pair&  \pageref{Dcornop}&  Lower-left corner of bounding box\\\hline
lrcorner\index{lrcorner?\texttt{lrcorner}}&  --&  picture\par path\par pen&  pair&  \pageref{Dcornop}&  Нижний правый угол охватывающей рамки\\\hline
%%lrcorner\index{lrcorner?\texttt{lrcorner}}&  --&  picture\par path\par pen&  pair&  \pageref{Dcornop}&  Lower-right corner of bounding box\\\hline
magentapart\index{magentapart?\texttt{magentapart}}&  --&  cmykcolor&  numeric&  \pageref{Dcmykprt}&  Извлечь второй компонент\\\hline
%%magentapart\index{magentapart?\texttt{magentapart}}&  --&  cmykcolor&  numeric&  \pageref{Dcmykprt}&  Extract the second component\\\hline
makepath\index{makepath?\texttt{makepath}}&  --&  pen&  path&  \pageref{Dmkpath}&  Замкнутый путь, охватывающий форму пера\\\hline
%%makepath\index{makepath?\texttt{makepath}}&  --&  pen&  path&  \pageref{Dmkpath}&  Cyclic path bounding the pen shape\\\hline
makepen\index{makepen?\texttt{makepen}}&  --&  path&  pen&  \pageref{Dmkpen}&  Многоугольное перо из выпуклой части узлов пути\\\hline
%%makepen\index{makepen?\texttt{makepen}}&  --&  path&  pen&  \pageref{Dmkpen}&  A polygonal pen made from the convex hull of the path knots\\\hline
mexp\index{mexp?\texttt{mexp}}&  --&  numeric&  numeric&  --&  Функция $\exp(x/256)$\\\hline
%%mexp\index{mexp?\texttt{mexp}}&  --&  numeric&  numeric&  --&  The function $\exp(x/256)$\\\hline
mlog\index{mlog?\texttt{mlog}}&  --&  numeric&  numeric&  --&  Функция $256\ln(x)$\\\hline
%%mlog\index{mlog?\texttt{mlog}}&  --&  numeric&  numeric&  --&  The function $256\ln(x)$\\\hline
\pl mod\index{mod?\texttt{mod}}&  --&  numeric&  numeric&  --&  Функция-остаток $l-r\lfloor l/r\rfloor$\\\hline
%%\pl mod\index{mod?\texttt{mod}}&  --&  numeric&  numeric&  --&  The remainder function $l-r\lfloor l/r\rfloor$\\\hline
normal\-deviate\index{normaldeviate?\texttt{normaldeviate}}&  --&  --&  numeric&  --&  Выбор случайного числа со средним~0 и стандартным отклонением~1\\\hline
%%normal\-deviate\index{normaldeviate?\texttt{normaldeviate}}&  --&  --&  numeric&  --&  Choose a random number with mean~0 and standard deviation~1\\\hline
not\index{not?\texttt{not}}&  --&  boolean&  boolean&  \pageref{Dnot}&  Логическое НЕ\\\hline
%%not\index{not?\texttt{not}}&  --&  boolean&  boolean&  \pageref{Dnot}&  Logical negation\\\hline
numeric\index{numeric?\texttt{numeric}}&  --&  любой&  boolean&  \pageref{Dnumop}&  Выражение числового типа?\\\hline
%%numeric\index{numeric?\texttt{numeric}}&  --&  any&  boolean&  \pageref{Dnumop}&  Is the expression of type numeric?\\\hline
oct\index{oct?\texttt{oct}}&  --&  string&  numeric&  --&  Интерпретировать строку как 8-ричное число\\\hline
%%oct\index{oct?\texttt{oct}}&  --&  string&  numeric&  --&  Interpret string as octal number\\\hline
odd\index{odd?\texttt{odd}}&  --&  numeric&  boolean&  --&  Ближайшее целое нечетное?\\\hline
%%odd\index{odd?\texttt{odd}}&  --&  numeric&  boolean&  --&  Is the closest integer odd or even?\\\hline
or\index{or?\texttt{or}}&  boolean&  boolean&  boolean&  \pageref{Dor}&  Логическое ИЛИ\\\hline
%%or\index{or?\texttt{or}}&  boolean&  boolean&  boolean&  \pageref{Dor}&  Logical inclusive or\\\hline
pair\index{pair?\texttt{pair}}&  --&  любой&  boolean&  \pageref{Dpairop}&  Выражение типа пара?\\\hline
%%pair\index{pair?\texttt{pair}}&  --&  any&  boolean&  \pageref{Dpairop}&  Is the expression of type pair?\\\hline
path\index{path?\texttt{path}}&  --&  любой&  boolean&  \pageref{Dpathop}&  Выражение типа путь?\\\hline
%%path\index{path?\texttt{path}}&  --&  any&  boolean&  \pageref{Dpathop}&  Is the expression of type path?\\\hline
pathpart\index{pathpart?\texttt{pathpart}}&  --&  picture&  path&  \pageref{Dpathpart}&  Компонент-путь нарисованной картинки\\\hline
%%pathpart\index{pathpart?\texttt{pathpart}}&  --&  picture&  path&  \pageref{Dpathpart}&  Path of a stroked picture component\\\hline
pen\index{pen?\texttt{pen}}&  --&  любой&  boolean&  \pageref{Dpenop}&  Выражение типа перо?\\\hline
%%pen\index{pen?\texttt{pen}}&  --&  any&  boolean&  \pageref{Dpenop}&  Is the expression of type pen?\\\hline
penoffset of\index{penoffset?\texttt{penoffset}}&  pair&  pen&  pair&  --&  Крайняя точка пера с  заданным направлением\\\hline
%%penoffset of\index{penoffset?\texttt{penoffset}}&  pair&  pen&  pair&  --&  Point on the pen furthest to the right of the given direction\\\hline
penpart\index{penpart?\texttt{penpart}}&  --&  picture&  pen&  \pageref{Dpenpart}&  Компонента-перо нарисованной картинки\\\hline
%%penpart\index{penpart?\texttt{penpart}}&  --&  picture&  pen&  \pageref{Dpenpart}&  Pen of a stroked picture component\\\hline
picture\index{picture?\texttt{picture}}&  --&  любой&  boolean&  \pageref{Dpictop}&  Выражение типа картинка?\\\hline
%%picture\index{picture?\texttt{picture}}&  --&  any&  boolean&  \pageref{Dpictop}&  Is the expression of type picture?\\\hline
point of\index{point of?\texttt{point of}}&  numeric&  path&  pair&  \pageref{Dpntof}&  Точка на пути с заданным значением времени\\\hline
%%point of\index{point of?\texttt{point of}}&  numeric&  path&  pair&  \pageref{Dpntof}&  Point on a path given a time value\\\hline
postcontrol of\index{postcontrol?\texttt{postcontrol}}&  numeric&  path&  pair&  --&  Первая управляющая точка Безье на отрезке пути, начинающимся в данное время\\\hline
%%postcontrol of\index{postcontrol?\texttt{postcontrol}}&  numeric&  path&  pair&  --&  First B\'ezier control point on path segment starting at the given time\\\hline
precontrol of\index{precontrol?\texttt{precontrol}}&  numeric&  path&  pair&  --&  Последняя управляющая точка Безье на отрезке пути, заканчивающимся в данное время\\\hline
%%precontrol of\index{precontrol?\texttt{precontrol}}&  numeric&  path&  pair&  --&  Last B\'ezier control point on path segment ending at the given time\\\hline
readfrom\index{readfrom?\texttt{readfrom}}&  --&  string&  string&  \pageref{Dreadfrom}&  Читать строку из файла\\\hline
%%readfrom\index{readfrom?\texttt{readfrom}}&  --&  string&  string&  \pageref{Dreadfrom}&  Read a line from file\\\hline
redpart\index{redpart?\texttt{redpart}}&  --&  color&  numeric&  \pageref{Drgbprt}&  Выделить первый компонент\\\hline
%%redpart\index{redpart?\texttt{redpart}}&  --&  color&  numeric&  \pageref{Drgbprt}&  Extract the first component\\\hline
reverse\index{reverse?\texttt{reverse}}&  --&  path&  path&  \pageref{Drevrse}&  путь в обратном `времени', конец меняется с началом\\\hline
%%reverse\index{reverse?\texttt{reverse}}&  --&  path&  path&  \pageref{Drevrse}&  `time'-reversed path, beginning swapped with ending\\\hline
rgbcolor\index{rgbcolor?\texttt{rgbcolor}}&  --&  любой&  boolean&  \pageref{Drcolrop}&  Выражение типа цвет?\\\hline
%%rgbcolor\index{rgbcolor?\texttt{rgbcolor}}&  --&  any&  boolean&  \pageref{Drcolrop}&  Is the expression of type color?\\\hline
rotated\index{rotated?\texttt{rotated}}&  picture\par path\par pair\par pen\par transform&  numeric&  picture\par path\par pair\par pen\par transform&  \pageref{Dtranop}&  Вращение (в градусах) против часовой стрелки\\\hline
%%rotated\index{rotated?\texttt{rotated}}&  picture\par path\par pair\par pen\par transform&  numeric&  picture\par path\par pair\par pen\par transform&  \pageref{Dtranop}&  Rotate counterclockwise a given number of degrees\\\hline
\pl round\index{round?\texttt{round}}&  --&  numeric\par pair&  numeric\par pair&  \pageref{Dround}&  округлить каждую компоненту до ближайшего целого\\\hline
%%\pl round\index{round?\texttt{round}}&  --&  numeric\par pair&  numeric\par pair&  \pageref{Dround}&  round each component to the nearest integer\\\hline
\pl rt\index{rt?\texttt{rt}}&  --&  numeric\par pair&  numeric\par pair&  \pageref{Drt}&  Правая сторона текущего пера, центрированного по данным координатам\\\hline
%%\pl rt\index{rt?\texttt{rt}}&  --&  numeric\par pair&  numeric\par pair&  \pageref{Drt}&  Right side of current pen when centered at given coordinate(s)\\\hline
scaled\index{scaled?\texttt{scaled}}&  picture\par path\par pair\par pen\par transform&  numeric&  picture\par path\par pair\par pen\par transform&  \pageref{Dtranop}&  Масштабируй все координаты в заданное число раз\\\hline
%%scaled\index{scaled?\texttt{scaled}}&  picture\par path\par pair\par pen\par transform&  numeric&  picture\par path\par pair\par pen\par transform&  \pageref{Dtranop}&  Scale all coordinates by the given amount\\\hline
scantokens\index{scantokens?\texttt{scantokens}}&  --&  string&  token sequence&  \pageref{Dscantokens}&  Преобрази строку в знак или последовательность знаков. Обеспечивает преобразование строки в число и т.~п.\\\hline
%%scantokens\index{scantokens?\texttt{scantokens}}&  --&  string&  token sequence&  \pageref{Dscantokens}&  Converts a string to a token or token sequence. Provides string to numeric conversion, etc.\\\hline
shifted\index{shifted?\texttt{shifted}}&  picture\par path\par pair\par pen\par transform&  pair&  picture\par path\par pair\par pen\par transform&  \pageref{Dtranop}&  Добавляет заданный сдвиг к каждой паре координат\\\hline
%%shifted\index{shifted?\texttt{shifted}}&  picture\par path\par pair\par pen\par transform&  pair&  picture\par path\par pair\par pen\par transform&  \pageref{Dtranop}&  Add the given shift amount to each pair of coordinates\\\hline
sind\index{sind?\texttt{sind}}&  --&  numeric&  numeric&  \pageref{Dsind}&  Синус угла в градусах\\\hline
%%sind\index{sind?\texttt{sind}}&  --&  numeric&  numeric&  \pageref{Dsind}&  Sine of an angle in degrees\\\hline
slanted\index{slanted?\texttt{slanted}}&  picture\par path\par pair\par pen\par transform&  numeric&  picture\par path\par pair\par pen\par transform&  \pageref{Dtranop}&  Применение трансформации-наклона, переводящей $(x,y)$ в $(x+sy,y)$, где~$s$ --- аргумент-число\\\hline
%%slanted\index{slanted?\texttt{slanted}}&  picture\par path\par pair\par pen\par transform&  numeric&  picture\par path\par pair\par pen\par transform&  \pageref{Dtranop}&  Apply the slanting transformation that maps $(x,y)$ into $(x+sy,y)$, where~$s$ is the numeric argument\\\hline
sqrt\index{sqrt?\texttt{sqrt}}&  --&  numeric&  numeric&  \pageref{Dsqrt}&  Квадратный корень\\\hline
%%sqrt\index{sqrt?\texttt{sqrt}}&  --&  numeric&  numeric&  \pageref{Dsqrt}&  Square root\\\hline
str\index{str?\texttt{str}}&  --&  suffix&  string&  \pageref{Dstr}&  Строковое представление суффикса\\\hline
%%str\index{str?\texttt{str}}&  --&  suffix&  string&  \pageref{Dstr}&  String representation for a suffix\\\hline
string\index{string?\texttt{string}}&  --&  любой&  boolean&  \pageref{Dstrgop}&  Выражение типа строка?\\\hline
%%string&  --&  any&  boolean&  \pageref{Dstrgop}&  Is the expression of type string?\\\hline
stroked\index{stroked?\texttt{stroked}}&  --&  любой&  boolean&  \pageref{Dstroked}&  Аргумент --- это нарисованная линия?\\\hline
%%stroked\index{stroked?\texttt{stroked}}&  --&  any&  boolean&  \pageref{Dstroked}&  Is argument a stroked line?\\\hline
subpath of\index{subpath?\texttt{subpath}}&  pair&  path&  path&  \pageref{Dsubpth}&  Часть пути для заданного диапазона времени\\\hline
%%subpath of\index{subpath?\texttt{subpath}}&  pair&  path&  path&  \pageref{Dsubpth}&  Portion of a path for given range of time values\\\hline
substring of\index{substring
of?\texttt{substring of}}&  pair&  string&  string&  \pageref{Dsubstr}&  Подстрока, ограниченная индексами\\\hline
%%substring of\index{substring of?\texttt{substring of}}&  pair&  string&  string&  \pageref{Dsubstr}&  Substring bounded by given indices\\\hline
textpart\index{textpart?\texttt{textpart}}&  --&  picture&  string&  \pageref{Dtextpart}&  Текст текстовой компоненты картинки\\\hline
%%textpart\index{textpart?\texttt{textpart}}&  --&  picture&  string&  \pageref{Dtextpart}&  Text of a textual picture component\\\hline
textual\index{textual?\texttt{textual}}&  --&  любой&  boolean&  \pageref{Dtextual}&  Аргумент --- это текст?\\\hline
%%textual\index{textual?\texttt{textual}}&  --&  any&  boolean&  \pageref{Dtextual}&  Is argument typeset text?\\\hline
\pl top\index{top?\texttt{top}}&  --&  numeric\par pair&  numeric\par pair&  \pageref{Dtop}&  Верх текущего пера, центрированного по заданным координатам\\\hline
%%\pl top\index{top?\texttt{top}}&  --&  numeric\par pair&  numeric\par pair&  \pageref{Dtop}&  Top of current pen when centered at the given coordinate(s)\\\hline
transform\index{transform?\texttt{transform}}&  --&  любой&  boolean&  \pageref{Dtrnfop}&  Аргумент типа трансформация?\\\hline
%%transform\index{transform?\texttt{transform}}&  --&  any&  boolean&  \pageref{Dtrnfop}&  Is the argument of type transform?\\\hline
transformed\index{transformed?\texttt{transformed}}&  picture\par path\par pair\par pen\par transform&  transform&  picture\par path\par pair\par pen\par transform&  \pageref{Dtrfrmd}&  Примени данную трансформацию ко всем координатам\\\hline
%%transformed\index{transformed?\texttt{transformed}}&  picture\par path\par pair\par pen\par transform&  transform&  picture\par path\par pair\par pen\par transform&  \pageref{Dtrfrmd}&  Apply the given transform to all coordinates\\\hline
ulcorner\index{ulcorner?\texttt{ulcorner}}&  --&  picture\par path\par pen&  pair&  \pageref{Dcornop}&  Верхний левый угол охватывающей рамки\\\hline
%%ulcorner\index{ulcorner?\texttt{ulcorner}}&  --&  picture\par path\par pen&  pair&  \pageref{Dcornop}&  Upper-left corner of bounding box\\\hline
uniform\-deviate\index{uniformdeviate?\texttt{uniformdeviate}}&  --&  numeric&  numeric&  --&  Случайное число от нуля до значения аргумента\\\hline
%%uniform\-deviate\index{uniformdeviate?\texttt{uniformdeviate}}&  --&  numeric&  numeric&  --&  Random number between zero and the value of the argument\\\hline
\pl unitvector\index{unitvector?\texttt{unitvector}}&  --&  pair&  pair&  \pageref{Duvec}&  Масштабируй вектор к длине~1\\\hline
%%\pl unitvector\index{unitvector?\texttt{unitvector}}&  --&  pair&  pair&  \pageref{Duvec}&  Rescale a vector so its length is~1\\\hline
unknown\index{unknown?\texttt{unknown}}&  --&  любой&  boolean&  \pageref{Dunknwn}&  Значение неизвестно?\\\hline
%%unknown\index{unknown?\texttt{unknown}}&  --&  any&  boolean&  \pageref{Dunknwn}&  Is the value unknown?\\\hline
urcorner\index{urcorner?\texttt{urcorner}}&  --&  picture\par path\par pen&  pair&  \pageref{Dcornop}&  Верхний правый угол охватывающей рамки\\\hline
%%urcorner\index{urcorner?\texttt{urcorner}}&  --&  picture\par path\par pen&  pair&  \pageref{Dcornop}&  Upper-right corner of bounding box\\\hline
\pl whatever\index{whatever?\texttt{whatever}}&  --&  --&  numeric&  \pageref{Dwhatev}&  Создай новую анонимную неизвестную\\\hline
%%\pl whatever\index{whatever?\texttt{whatever}}&  --&  --&  numeric&  \pageref{Dwhatev}&  Create a new anonymous unknown\\\hline
xpart\index{xpart?\texttt{xpart}}&  --&  pair\par transform&  number&  \pageref{Dxprt}&  $x$ или $t_x$ компонента\\\hline
%%xpart\index{xpart?\texttt{xpart}}&  --&  pair\par transform&  number&  \pageref{Dxprt}&  $x$ or $t_x$ component\\\hline
xscaled\index{xscaled?\texttt{xscaled}}&  picture\par path\par pair\par pen\par transform&  numeric&  picture\par path\par pair\par pen\par transform&  \pageref{Dtranop}&  Масштабируй все координаты $x$ в заданное число раз\\\hline
%%xscaled\index{xscaled?\texttt{xscaled}}&  picture\par path\par pair\par pen\par transform&  numeric&  picture\par path\par pair\par pen\par transform&  \pageref{Dtranop}&  Scale all $x$ coordinates by the given amount\\\hline
xxpart\index{xxpart?\texttt{xxpart}}&  --&  transform&  number&  \pageref{Dtrprt}&  $t_{xx}$ в матрице трансформации\\\hline
%%xxpart\index{xxpart?\texttt{xxpart}}&  --&  transform&  number&  \pageref{Dtrprt}&  $t_{xx}$ entry in transformation matrix\\\hline
xypart\index{xypart?\texttt{xypart}}&  --&  transform&  number&  \pageref{Dtrprt}&  $t_{xy}$ в матрице трансформации\\\hline
%%xypart\index{xypart?\texttt{xypart}}&  --&  transform&  number&  \pageref{Dtrprt}&  $t_{xy}$ entry in transformation matrix\\\hline
yellowpart\index{yellowpart?\texttt{yellowpart}}&  --&  cmykcolor&  numeric&  \pageref{Dcmykprt}&  Выделить третью компоненту\\\hline
%%yellowpart\index{yellowpart?\texttt{yellowpart}}&  --&  cmykcolor&  numeric&  \pageref{Dcmykprt}&  Extract the third component\\\hline
ypart\index{ypart?\texttt{ypart}}&  --&  pair\par transform&  number&  \pageref{Dyprt}&  Компонента $y$ или $t_y$\\\hline
%%ypart\index{ypart?\texttt{ypart}}&  --&  pair\par transform&  number&  \pageref{Dyprt}&  $y$ or $t_y$ component\\\hline
yscaled\index{yscaled?\texttt{yscaled}}&  picture\par path\par pair\par pen\par transform&  numeric&  picture\par path\par pair\par pen\par transform&  \pageref{Dtranop}&  Масштабируй все координаты $y$ в заданное число раз\\\hline
%%yscaled\index{yscaled?\texttt{yscaled}}&  picture\par path\par pair\par pen\par transform&  numeric&  picture\par path\par pair\par pen\par transform&  \pageref{Dtranop}&  Scale all $y$ coordinates by the given amount\\\hline
yxpart\index{yxpart?\texttt{yxpart}}&  --&  transform&  number&  \pageref{Dtrprt}&  $t_{yx}$ в матрице трансформации\\\hline
%%yxpart\index{yxpart?\texttt{yxpart}}&  --&  transform&  number&  \pageref{Dtrprt}&  $t_{yx}$ entry in transformation matrix\\\hline
yypart\index{yypart?\texttt{yypart}}&  --&  transform&  number&  \pageref{Dtrprt}&  $t_{yy}$ в матрице трансформации\\\hline
%%yypart\index{yypart?\texttt{yypart}}&  --&  transform&  number&  \pageref{Dtrprt}&  $t_{yy}$ entry in transformation matrix\\\hline
zscaled\index{zscaled?\texttt{zscaled}}&  picture\par path\par pair\par pen\par transform&  pair&  picture\par path\par pair\par pen\par transform&  \pageref{Dtranop}&  Вращать и масштабировать все координаты так, что $(1,0)$ становится заданной парой, т.~е. произвести комплексное умножение.\\\hline
%%zscaled\index{zscaled?\texttt{zscaled}}&  picture\par path\par pair\par pen\par transform&  pair&  picture\par path\par pair\par pen\par transform&  \pageref{Dtranop}&  Rotate and scale all coordinates so that $(1,0)$ is mapped into the given pair; i.e., do complex multiplication.\\\hline
\end{longtable}
